\section{Künstliche Intelligenz}
\label{sec:ai}

In diesem Abschnitt sind meine Notizen zu künstlicher Intelligenz zu finden.

\begin{itemize}
    \item Künstliche Intelligenz ist ein Teilgebiet der Informatik und beschäftigt sich mit maschinellem Lernen \citep{ai-wikipedia}.
    \item KI-Ethik: Bei der KI-Ethik handelt es sich um eine Reihe von Richtlinien, die uns einen Leitfaden bei der Entwicklung und der Nutzung der Ergebnisse künstlicher Intelligenz an die Hand geben. \citep{ibm}
\end{itemize}

\subsection{Vor- und Nachteile} \citep{GitHub}
\begin{itemize}
    \item[+] macht arbeit einfacher und effizienter
    \item[+] übernimmt monotone Aufgaben
    \item[+] vereinfacht nachschlagen
    \item[+] kann Kreativität fördern
    \item[-] falsche informationen
    \item[-] muss richtig genutzt werden
\end{itemize}

\bigskip

\subsection{Training einer KI } \citep{KI-Training}
\textbf{maschinelles lernen}--> Computer kann automatisiert lernen, sich verbessern, fähigkeiten verfeinern.
Algorithmen verwendet, die Beziehungen zwischen Variablen (d. h. Muster) entdecken und dann aus diesen Lektionen lernen, je mehr Daten sie erhalten.\\
\textbf{deep lerning}--> speziellere technik des maschinelles lernens. nachahmen des menschlichen gehirn bei der verarbeitung von daten. neuronale netze.\\


\textbf{training}--> 3 stufen:
\begin{description}
    \item[1. Training] Daten zuführen; Algorithmus die Daten analysieren und bessere Vorhersagen treffen
    \item[überwachtes lernen] von eine person stellt geeignete kennzeichnungen für die eingabedaten bereit, damit der computer keine verwchslungen/fehler macht
    \item[unüberwachtes lernen] 3 arten--> Clustering, Association Rule Mining (Assoziationsregel-Mining) und Ausreißererkennung.
    \item[clustering]  nicht gelabelte Daten nach bestimmten Kriterien zusammenzufassen. Die entsprechenden Daten können auf Grundlage von Ähnlichkeiten oder Unterschieden gruppiert werden, damit  bestimmte Datenpunkte zu Gruppen zusammengefasst werden können. Diese Art des unüberwachten Lernens ist nützlich für die Marktsegmentierung.
    \item[Assoziationsregel-Mining]  werden die Daten etwas anders betrachtet, mit der Absicht, Beziehungen zwischen Datenpunkten zu finden. Diese Art des unüberwachten Lernens ist nützlich, um die Beziehungen zwischen verschiedenen Gruppen von Elementen zu analysieren und herauszufinden, welche Kombinationen mit größerer Wahrscheinlichkeit zusammen auftreten.
    \item[aussenseitererkennung] Datenpunkte zu finden, die außerhalb Grenzen liegen. Diese Art des unüberwachten Lernens ist auch hilfreich, um Anomalien in Datensätzen zu finden, die möglicherweise zur Erkennung von ungewöhnlichem oder betrügerischem Verhalten führen.
    \item[2. Validierung]Validierungstest, bei dem bewertet wird, wie die Modelle bei Daten abschneiden, die das Modell noch nicht gesehen hat.--> training weiter oder nicht
    \item[3. Test] Geben Sie der KI einen Datensatz, der keine Tags oder Ziele enthält (diese haben ihr bisher bei der Interpretation der Daten geholfen). Nachdem Sie Ihre KI auf unstrukturierte Informationen trainiert haben, ist es an der Zeit, sie auf die Probe zu stellen.
    Je genauer die Entscheidungen sind, die Ihre künstliche Intelligenz treffen kann, desto besser sind Sie vorbereitet, wenn sie in Betrieb geht. Allerdings müssen Sie genauer hinschauen, ob Sie auch eine 100-prozentige Genauigkeit erreichen.
    \end{description}


\bigskip

\subsection{Verwendung der KI}
wie, wo und für was wird KI verwendet

wie KI funktioniert
