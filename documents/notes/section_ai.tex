\section{Künstliche Intelligenz}
\label{sec:ai}

In diesem Abschnitt sind meine Notizen zu künstlicher Intelligenz zu finden.

\begin{itemize}
    \item Künstliche Intelligenz ist ein Teilgebiet der Informatik und beschäftigt sich mit maschinellem Lernen \citep{ai-wikipedia}.
    \item KI-Ethik: Bei der KI-Ethik handelt es sich um eine Reihe von Richtlinien, die uns einen Leitfaden bei der Entwicklung und der Nutzung der Ergebnisse künstlicher Intelligenz an die Hand geben. \citep{ibm}
\end{itemize}

Vor- und Nachteile \citep{GitHub}:
\begin{itemize}
    \item[+] macht arbeit einfacher und effizienter
    \item[+] übernimmt monotone Aufgaben
    \item[+] vereinfacht nachschlagen
    \item[+] kann Kreativität fördern
    \item[-] falsche informationen
    \item[-] muss richtig genutzt werden
\end{itemize}

\bigskip

Training einer KI \citep{KI-Training}:
\textbf{maschinelles lernen}--> Computer kann automatisiert lernen, sich verbessern, fähigkeiten verfeinern.
Algorithmen verwendet, die Beziehungen zwischen Variablen (d. h. Muster) entdecken und dann aus diesen Lektionen lernen, je mehr Daten sie erhalten.\\
\textbf{deep lerning}--> speziellere technik des maschinelles lernens. nachahmen des menschlichen gehirn bei der verarbeitung von daten. neuronale netze.\\

\textbf{training}--> 3 stufen:
\begin{description}
    \item[1. Training] Daten zuführen; Algorithmus die Daten analysieren und bessere Vorhersagen treffen
    \item[2. Validierung] Nützlich für Verfahren
    \item[3. Test] Nützlich für Vokabeln
    \end{description}


\bigskip

wie KI funktioniert
