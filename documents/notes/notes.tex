\documentclass{article}

\usepackage[ngerman]{babel}
\usepackage[utf8]{inputenc}
\usepackage[T1]{fontenc}
\usepackage{hyperref}
\usepackage{csquotes}

\usepackage[
    backend=biber,
    style=apa,
    sortlocale=de_DE,
    natbib=true,
    url=false,
    doi=false,
    sortcites=true,
    sorting=nyt,
    isbn=false,
    hyperref=true,
    backref=false,
    giveninits=false,
    eprint=false]{biblatex}
\addbibresource{../references/bibliography.bib}

\title{Notizen zum Projekt Data Ethics}
\author{Vanessa Tagliente}
\date{\today}

\begin{document}
\maketitle

\abstract{
    Dieses Dokument ist eine Sammlung von Notizen zu dem Projekt. Die Struktur innerhalb des
    Projektes ist gleich ausgelegt wie in der Hauptarbeit, somit kann hier einfach geschrieben
    werden, und die Teile die man verwenden möchte, kann man direkt in die Hauptdatei ziehen.
}

\tableofcontents

\section{Künstliche Intelligenz}
\label{sec:ai}

In diesem Abschnitt sind meine Notizen zu künstlicher Intelligenz zu finden.

\begin{itemize}
    \item Künstliche Intelligenz ist ein Teilgebiet der Informatik und beschäftigt sich mit maschinellem Lernen \citep{ai-wikipedia}.
    \item KI-Ethik: Bei der KI-Ethik handelt es sich um eine Reihe von Richtlinien, die uns einen Leitfaden bei der Entwicklung und der Nutzung der Ergebnisse künstlicher Intelligenz an die Hand geben. \citep{ibm}
\end{itemize}

\subsection{Vor- und Nachteile} \citep{GitHub}
\begin{itemize}
    \item[+] macht arbeit einfacher und effizienter
    \item[+] übernimmt monotone Aufgaben
    \item[+] vereinfacht nachschlagen
    \item[+] kann Kreativität fördern
    \item[-] falsche informationen
    \item[-] muss richtig genutzt werden
\end{itemize}

\bigskip

\subsection{Training einer KI } \citep{KI-Training}
\textbf{maschinelles lernen}--> Computer kann automatisiert lernen, sich verbessern, fähigkeiten verfeinern.
Algorithmen verwendet, die Beziehungen zwischen Variablen (d. h. Muster) entdecken und dann aus diesen Lektionen lernen, je mehr Daten sie erhalten.\\
\textbf{deep lerning}--> speziellere technik des maschinelles lernens. nachahmen des menschlichen gehirn bei der verarbeitung von daten. neuronale netze.\\


\textbf{training}--> 3 stufen:
\begin{description}
    \item[1. Training] Daten zuführen; Algorithmus die Daten analysieren und bessere Vorhersagen treffen
    \item[2. Validierung] Nützlich für Verfahren
    \item[3. Test] Nützlich für Vokabeln
    \end{description}


\bigskip

\subsection{Verwendung der KI}
wie, wo und für was wird KI verwendet

wie KI funktioniert


\section{Umgang mit daten}
\label{sec:Data}

In diesem Abschnitt sind meine Notizen zu umgang mit Daten zu finden.

\subsection{Datenumgang im internet}
\begin{itemize}
    \item sichere passwörter
    \item nie überall das gleiche passwort und diese erneuern
    \item agb von diest-/plattformanbiern lesen
    \item antivirusprogramme und firewall verwenden
    \item programme und browser nur von sicheren (offiziellen und seriösen) seiten herunterladen
    \item cookieeinstellungen löschen/überprüfen
    \item cache-speicher und browserverlauf regelmässig löschen
    \item sicherheitseinstellungen von suchmaschinen prüfen
    \item passwörter und pins für mobiltelefon, computer, tablet usw verwenden
    \item im öffentlichen wlan keine sensiblen daten verwenden
    \item sichergehen dass datenverkehr verschlüsselt
    \item bluetooth ausschalten
\end{itemize}
\citep{datenschutz}


\printbibliography

\end{document}
