\section{Umgang mit daten}
\label{sec:Data}

In diesem Abschnitt sind meine Notizen zu umgang mit Daten zu finden.

\subsection{Datenumgang im internet}
\begin{itemize}
    \item sichere passwörter
    \item nie überall das gleiche passwort und diese erneuern
    \item agb von diest-/plattformanbiern lesen
    \item antivirusprogramme und firewall verwenden
    \item programme und browser nur von sicheren (offiziellen und seriösen) seiten herunterladen
    \item cookieeinstellungen löschen/überprüfen
    \item cache-speicher und browserverlauf regelmässig löschen
    \item sicherheitseinstellungen von suchmaschinen prüfen
    \item passwörter und pins für mobiltelefon, computer, tablet usw verwenden
    \item im öffentlichen wlan keine sensiblen daten verwenden
    \item sichergehen dass datenverkehr verschlüsselt
    \item bluetooth ausschalten
\end{itemize}
\citep{datenschutz}

\subsection{ethik}
-  Bei der KI-Ethik handelt es sich um eine Reihe von Richtlinien, die uns einen Leitfaden bei der Entwicklung und der Nutzung der Ergebnisse künstlicher Intelligenz an die Hand geben.\\
-  ethik in der experimentiellen forschung und entwicklung von algorithmen:\\
- respekt für personen: einzelperson, die an experiment teilnimmt muss sich risiken bewusst sein und kann jederzeit aussteigen. gehandicapte personen sollte man schützen\\
- es sollte kein schaden zugefügt werden. ki kann trotzdem bestehende vorurteile verstärken\\
- privatsphäre schützen

\citep{ibm}
