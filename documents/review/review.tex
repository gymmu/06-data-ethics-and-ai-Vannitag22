\documentclass{article}

\usepackage[ngerman]{babel}
\usepackage[utf8]{inputenc}
\usepackage[T1]{fontenc}
\usepackage{hyperref}
\usepackage{csquotes}

\usepackage[
    backend=biber,
    style=apa,
    sortlocale=de_DE,
    natbib=true,
    url=false,
    doi=false,
    sortcites=true,
    sorting=nyt,
    isbn=false,
    hyperref=true,
    backref=false,
    giveninits=false,
    eprint=false]{biblatex}
\addbibresource{../references/bibliography.bib}

\title{Review des Papers "Ethik im Umgang mit Daten" von \dots}
\author{Yeva Skotar}
\date{\today}

\begin{document}
\maketitle

\abstract{
    Dies ist ein Review der Arbeit zum Thema Ethik im Umgang mit Daten von <Yeva Skotar>.
    
    Das Review wurde von Vanessa Tagliente für Yeva Skotar erstellt.

}

\section{Inhalt}
Die Einleitung ist schön ausführlich geschrieben und bildet gute Übergänge zum eigentlichen Thema.

Meines Wissens nach stimmen die genannten Aussagen, und  die Themen werden gut und verständlich erklärt. Es wurden - denke ich - verlässliche Quellen verwendet.

Die ethischen Probleme werden auf den Punkt gebracht und die Rolle der Mitarbeiter wird ebenfalls erklärt. Dass man die Mitarbeiter erwähnt finde ich gut, da diese Menschen sehr viel arbeiten, aber fast niemand verschwendet einen Gedanken an sie und sie zu wenig gewertschätzt werden.
Im Fazit wird das Thema nochmals zusammengefasst.

Man hätte eventuell Vor- und Nachteile erwähnen können oder wie man am besten mit den Daten umgeht.

\section{Funktionen und Rechtschreibung}
Es wurden viele Latex-Funktionen wie kursiver oder fett geschriebener Text verwendet. In der Arbeit wurden Bilder 
verwendet, welche dem erklärten Thema eine grobe Übersicht verleihen.

Die Arbeit hat allerdings einige Rechtschreib- und Grammatikfehler. Diese haben ich aber nicht korrigiert.
Man kann bei den Fehler aber ein Auge zudrücken, da der Autor die deutsche Sprache noch nicht perfekt beherrscht.
\bigskip

Insgesammt finde ich, ist es eine gute Arbeit und alles wird verständlich erklärt.


\end{document}
