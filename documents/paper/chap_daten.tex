\chapter{Umgang mit Daten}
\label{chap:daten}

Wie soll man mit seinen persönlichen Daten umgehen? Was kann man tunn damit diese nicht ins Internet gelangen? Wie geht die KI mit den Daten um?
Das  sind alles Fragen, die vermutlich jedem schon mal durch den Kopf gegangen sind. Und genau diese werden hier bearebeitet.

\section{Umgang mit Daten im Internet}
So können sie ihre Daten vor dem Internet schützen.

\begin{itemize}
    \item Sparsam mit persönlichen und sensiblen Daten umgehen. Nie mehr als nötig angeben
    \item Sichere und mehrere Passwörter verwenden und diese erneuern
    \item AGB von Diest-/Plattformanbiern lesen
    \item Antivirusprogramme und Firewalls verwenden
    \item Programme und Browser nur von sicheren (offiziellen und seriösen) Seiten herunterladen
    \item Cookieeinstellungen löschen/überprüfen
    \item Cache-Speicher und Browserverlauf regelmässig löschen
    \item Sicherheitseinstellungen von Suchmaschinen prüfen
    \item Passwörter und Pins für Mobiltelefon, Computer, Tablet, usw. verwenden
    \item Im öffentlichen W-lan keine sensiblen Daten verwenden
    \item Sichergehen, dass der Datenverkehr verschlüsselt ist
    \item Bluetooth ausschalten
\end{itemize}
\citep{datenschutz}

\section{Ethik}
Bei der KI-Ethik handelt es sich um eine Reihe von Richtlinien, die uns einen Leitfaden bei der Entwicklung und der Nutzung der Ergebnisse künstlicher Intelligenz geben.
\bigskip
Ethik in der experimentiellen Forschung und Entwicklung von Algorithmen:\\
- Die Personen sollen respektvoll behandelt werden. Die Einzelperson, welche an den Experimenten teilnimmt muss sich Risiken bewusst sein und kann jederzeit aussteigen. Gehandicapte Personen sollte man schützen.\\
- Es sollte kein Schaden erzeugt werden. KI kann trotz bestimmten Algorithmen, die das verhindern sollten, bestehende Vorurteile verstärken.\\
- Die Privatsphäre muss geschützt werden.\\
\citep{ibm}
