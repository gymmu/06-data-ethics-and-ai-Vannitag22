\documentclass{report}

\usepackage[ngerman]{babel}
\usepackage[utf8]{inputenc}
\usepackage[T1]{fontenc}
\usepackage{hyperref}
\usepackage{csquotes}
\usepackage[a4paper]{geometry}
\usepackage{graphicx} %Um externe Bilddateien einfügen zu können.
\usepackage{float}
\usepackage{caption}

\usepackage[
    backend=biber,
    style=apa,
    sortlocale=de_DE,
    natbib=true,
    url=false,
    doi=false,
    sortcites=true,
    sorting=nyt,
    isbn=false,
    hyperref=true,
    backref=false,
    giveninits=false,
    eprint=false]{biblatex}
\addbibresource{../references/bibliography.bib}


\title{Ethik im Umgang mit Daten}
\author{Vanessa Tagliente}
\date{\today}

\begin{document}

\maketitle

\abstract{
    Dies ist eine Vorlage für eine Maturarbeit in der Informatik am Gymnasium Muttenz. Sie dient dazu, die Arbeit schnell und einfach zu starten und sollte einen guten Überblick über die Arbeit bieten.
}

\tableofcontents

\chapter{Einleitung}

In diesem Projekt befassen wir uns mit der Künstlichen Inteligenz (Kurz KI).
In diesem Dokument werden folgende Fragen bearbeitet: Was ist KI, Was sind die Vor- und Nachteile, Wie funktioniert die KI und wie wird sie trainiert.

\bigskip

Ich kann weitere Kapitel auch importieren.

\chapter{Training einer KI}
\label{chap:training}

In diesem Kapitel werden wir lernen, wieso und wie eine KI funktioniert.

\begin{figure}[h]
    \centering
    \includegraphics[width=0.5\textwidth]{ki-aufbau.jpg}
    \caption{Aufbau der KI}
    \label{fig:ki-aufbau}
\end{figure}

\section{Maschinelles Lernen}
Die KI verwendet \textbf{Maschinelles Lernen}. Damit kann der Computer automatisiert lernen. Er kann sich dabei ständig verbessern und seine Fähigkeiten verfeinern.
Für diese Methode werden Algorithmen verwendet, die Beziehungen zwischen Variablen (d. h. Muster) entdecken und dann aus diesen Lektionen lernen. Je mehr Daten sie erhalten, desto effizienter wird die KI.

\subsection{Deep Learning}
Beim \textbf{Deep Learning}-Verfahren werden \textbf{neuronale Netze} verwendet.
Neuronale Netze ahmen die Funktionsweise des menschliche Gehirns durch Algorithmen nach. Sie können aus Datenquellen Informationen und Muster erkennen und diese auf unbekannte Daten anwenden.
Das \textit{Deep Learning} wird dazu verwendet, um Bilder zu erkennen, Texte zu verstehen und Entscheidungen genauer zu tätigen.

\section{Training in drei Schritten}
Das Training einer KI ist in folgende drei Schritte aufgeteilt.

\subsection{Training}
Im ersten Schritt wird der Computer mit Daten gefüttert. Algorithmen können die Daten analysieren um bessere Vorhersagen zu treffen und die Genauigkeit von denen zu bewerten. Dafür wird das maschinelle Lernen angewandt.
Für das KI-Training gibt es zwei Methoden: überwachtes und unüberwachtes Lernen.
\\
\textbf{Überwachtes Lernen} Die Eingebedaten werden von einer Person geeignet gekennzeichnet, damit dem Computer keine Verwechslungen unterlaufen.
\\
\textbf{Unüberwachtes Lernen:} Beim unüberwachten Lernen gibt es wiederum drei Arten:  Clustering, Association Rule Mining (Assoziationsregel-Mining) und Ausreissererkennung. Beim \textit{Clustering} werden nicht gelabelte Daten nach bestimmten Kriterien zusammengefasst, indem die KI Ähnlichkeiten und Unterschiede mit den bereits gelabelten Daten sucht.
Durch das \textit{Association Rule Mining} werden die Daten etwas anderes betrachtet, mit der Absicht, Beziehungen zwischen Datenpunkten und verschiedenen Elementgruppen zu finden.
Die \textit{Ausreissererkennunng} wird verwendet, um Anomalien in Datensätzen zu finden, die außerhalb bestimmter Grenzen liegen.

\subsection{Validierung}
Im Valiedierungstest wird die Genauigkeit der KI mit den Testdaten verglichen. Damit kann man feststellen, ob die KI weiteres Training braucht oder bereit ist.

\subsection{Test}
Der KI werden unstrukturierte Datensätze (d.h. ohne Tags oder Ziele) gegeben. Nachdem sie darauf trainiert ist, wird sie auf die Probe gestellt. Je genauer die Entscheidungen sind, desto besser sind Sie vorbereitet, wenn sie in Betrieb geht.

\section{Quellen} \citep{KI-Training}, \citep{neuronale-netze}

\chapter{Umgang mit Daten}
\label{chap:daten}

Wie soll man mit seinen persönlichen Daten umgehen? Was kann man tunn damit diese nicht ins Internet gelangen? Wie geht die KI mit den Daten um?
Das  sind alles Fragen, die vermutlich jedem schon mal durch den Kopf gegangen sind. Und genau diese werden hier bearebeitet.

\section{Umgang mit Daten im Internet}
So können sie ihre Daten vor dem Internet schützen.

\begin{itemize}
    \item Sparsam mit persönlichen und sensiblen Daten umgehen. Nie mehr als nötig angeben
    \item Sichere und mehrere Passwörter verwenden und diese erneuern
    \item AGB von Diest-/Plattformanbiern lesen
    \item Antivirusprogramme und Firewalls verwenden
    \item Programme und Browser nur von sicheren (offiziellen und seriösen) Seiten herunterladen
    \item Cookieeinstellungen löschen/überprüfen
    \item Cache-Speicher und Browserverlauf regelmässig löschen
    \item Sicherheitseinstellungen von Suchmaschinen prüfen
    \item Passwörter und Pins für Mobiltelefon, Computer, Tablet, usw. verwenden
    \item Im öffentlichen W-lan keine sensiblen Daten verwenden
    \item Sichergehen, dass der Datenverkehr verschlüsselt ist
    \item Bluetooth ausschalten
\end{itemize}
\citep{datenschutz}

\section{Ethik}
Bei der KI-Ethik handelt es sich um eine Reihe von Richtlinien, die uns einen Leitfaden bei der Entwicklung und der Nutzung der Ergebnisse künstlicher Intelligenz geben.
\bigskip
Ethik in der experimentiellen Forschung und Entwicklung von Algorithmen:\\
- Die Personen sollen respektvoll behandelt werden. Die Einzelperson, welche an den Experimenten teilnimmt muss sich Risiken bewusst sein und kann jederzeit aussteigen. Gehandicapte Personen sollte man schützen.\\
- Es sollte kein Schaden erzeugt werden. KI kann trotz bestimmten Algorithmen, die das verhindern sollten, bestehende Vorurteile verstärken.\\
- Die Privatsphäre muss geschützt werden.\\
\citep{ibm}


\chapter{Fazit}
\label{chap:fazit}

\section{Vor- und Nachteile}
Wie alles, hat die KI ihre Vor- und Nachteile. Welche, werden hier aufgezählt.

\begin{itemize}
    \item[+] Die KI macht die Arbeit einfacher und effizienter
    \item[+] Sie übernimmt monotone Aufgaben
    \item[+] und vereinfacht das Nachschlagen eines Themas oder Begriffen
    \item[+] Sie kann die Kreativität fördern
    \item[]
    \item[-] Es können aber auch falsche Informationen angegeben werde, je nach dem, wie sie trainiert und was gefragt wurde.
    \item[-] Sie muss richtig genutzt werden, um die gewollten Ergebnisse zu erzielen
\end{itemize}

\citep{GitHub}

\input{chap_methode.tex}

\section{Etwas mit Quellen}

Etwas mit Änderung hier am Ende.

Wenn ich eine Quelle zitieren möchte, kann ich das ganze einfach am Ende des Satzes machen \citep{example}. Oder wie \citet{example} sagt, auch mitten im Text.

\printbibliography

\end{document}
