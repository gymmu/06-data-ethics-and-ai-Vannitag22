\documentclass{article}

\usepackage[ngerman]{babel} %Sprachelemente wie z.B. Silbentrennung
\usepackage[utf8]{inputenc} %Damit Umlaute und Sonderzeichen direkt eingegeben werden können
\usepackage[T1]{fontenc}
\usepackage{hyperref}
\usepackage{csquotes}
\usepackage[a4paper]{geometry}
\usepackage{xcolor} %Farben
\usepackage{setspace} %Zeilenabstand definieren
\usepackage{hyperref} %Verlinkungen
\usepackage{multicol} %Mehrere Spalten
\usepackage{graphicx} %Um externe Bilddateien einfügen zu können.
\usepackage{float}
\usepackage{caption}

\usepackage[
    backend=biber,
    style=apa,
    sortlocale=de_DE,
    natbib=true,
    url=false,
    doi=false,
    sortcites=true,
    sorting=nyt,
    isbn=false,
    hyperref=true,
    backref=false,
    giveninits=false,
    eprint=false]{biblatex}
    \addbibresource{../references/bibliography.bib}
    
%\renewcommand*\familydefault{\sfdefault} %Falls keine Serifen gewünscht

% Definiere neue Theoreme. Diese können als Baublöcke für Aufgaben oder ähnliches verwendet werden.
\newtheorem{theorem}{Theorem}[section]
\newtheorem{rem}[theorem]{Remark}
\newtheorem{ex}[theorem]{Übung}

% Setze den Zeilenabstand
\onehalfspacing %1.5 facher Zeilenabstand

% Definiere automatische Übersetzungen.
\addto{\captionsngerman}{
	\renewcommand\refname{Literaturverzeichnis} %Name für Quellenverzeichnis
}

\title{Eine kurze Einführung in \LaTeX}
\author{C. Geissmann}

% Hier wird das Dokument gestartet. Alles was darin ist, wird im Dokument abgedruckt.
\begin{document}

% Hier wird die Titelseite gestaltet.
\begin{titlepage}
    \makeatletter % Das hier brauchen wir damit wir spezielle Befehle wie \@author verwenden können.
	\begin{center}
		{\scshape Gymnasium Muttenz} \vspace{0.5cm}

		 Informatik 2023/2024\vspace{5.5cm}

		{\huge\bfseries \@title}

		\vspace{2cm}

		{\Large\itshape \@author}

        \vspace{2cm}

        Version vom: \@date
	\end{center}
    
    \makeatother % Wir müssen das @ wieder schliessen, damit der Rest ganz normal funktioniert.
\end{titlepage}
\newpage

\setcounter{page}{1} %Seitennummern nach Titelseite mit 1 beginnen
\tableofcontents %Inhaltsverzeichnis einblenden
\clearpage %Seitenumbruch


\input{01_einführung.tex} \newpage
\input{02_grundlagen.tex} \newpage
\input{03_listen.tex} \newpage
\input{04_bilder.tex} \newpage
\input{05_tabellen.tex} \newpage
\input{06_mathe.tex} \newpage
\input{07_aufbau.tex} \newpage

\end{document}